\documentclass[12 pt]{article}
%\usepackage[frenchb]{babel}
%\usepackage{draftcopy}
%\usepackage{url}
%%% eurosym provides \euro
%\usepackage{eurosym}
%%% Use \pounds for GBP
%\usepackage[utf8]{inputenc}
%\usepackage[T1]{fontenc}
\usepackage{fullpage}
\usepackage{svg}
\svgpath{{svg-inkscape/}}
\usepackage{graphicx}
\DeclareGraphicsExtensions{.pdf,.png,.jpg}
\usepackage{fontspec}
\setmainfont{TeX Gyre Heros}
\setsansfont{TeX Gyre Heros}

\pagestyle{empty}
\parskip=8 pt
\raggedright

\newcommand{\bodyimage}[1]{\rightline{\includesvg[width=.7\textwidth]{images/{#1}}}}
\newcommand{\separator}{\bigskip\hrule\bigskip}
\newcommand{\monopolesection}[1]{%
  \vbox{\bigskip\hrule
  \section{#1}%
  \separator}
}

\begin{document}

\begin{minipage}[t]{0.5\linewidth}
  \vspace{0pt}
  \includesvg[width=1\textwidth]{images/user-manual-hero.svg}  
\end{minipage}\begin{minipage}[t]{0.5\linewidth}
  \vspace{0pt}
  \begin{flushright}
    \Huge
    No O1 manual
  \end{flushright}
\end{minipage}

\includegraphics[width=\textwidth]{images/monopole_no_o1_belt_steering.png}

\separator

Before riding your MONoPOLE bike, please read and understand the
following safety information.

If you lack professional-level bicycle mechanic skills or feel unsure,
consult a qualified mechanic. Failure to follow the instructions in
this manual may result in serious injury.

\separator

\monopolesection{Original MONoPOLE parts}

\subsection{Source parts}

You must source these parts from MONoPOLE, or get in touch about the
precise specification for the parts which are standard components and
available from third party suppliers.

\subsection{Steering mechanism}

All parts of the steering mechanism must be original and conform to
MONoPOLE specifications. This includes the fork, cable clamps, cable
stops, cable stop bolts, frame rubber grommets, steering belt, front
headset, long steerer unit, and rear headset.

\subsection{Cargo racks}

The front cargo racks and all parts needed to attach them to the
bicycle/frame must be original to the MONoPOLE specification. This
includes the rack itself and the mounting hardware.


\monopolesection{Steering}

\subsection{Belt tension}

Using the official Gates® Carbon Drive tension measuring tool (Gates
art.nr. S8100030) measure the steering belt tension. It needs to be
within the range of 7.5 to 10. If the tension falls outside this
range, do not ride the bike. Instead, adjust the belt tension as
described here below.

\subsection{Belt excenter adjustment}

The marking on the long steerer tube excentric insert indicates the
position of the excenter. The more it is turned forwards, the more the
belt will be tensioned. Turn the excentric one step at a time and
measure the tension in between. Keep repeating until the correct value
is achieved.

\bodyimage{2_1_1_excenter_mark.svg}

\subsection{Important}

When adjusting the belt tension, turn the eccentric insert towards the
non-drive side of the bike.

\bodyimage{2_1_2_belt_excenter_adjustment.svg}

\subsection{Belt securing blocks}

Make sure the belt securing blocks are installed on both
sprockets. Make sure they are installed using a medium strength (non
permanent) threadlock solution. Make sure the bolts locking the belt
to the sprocket are only tightened until the securing blocks sit flush
against the belt. As soon as the belt starts to deform, you have
tightened the bolts too much. Back up the preload until the belt is no
longer deformed by the bolt tension. Let the threadlock dry to the
manufacturers specification before you use the bike!

\bodyimage{2_3_belt_securing_block.svg}

\subsection{Long steerer headset preload}

Every No O1 comes with a preinstalled steerer clamp to allow for stem
changes and adjustments without having to retighten the long headtube
headset. If, for whatever reason, you need to remove the steerer
clamp, follow the instructions under 2.6 and 2.7 to retension the
headset.

\bodyimage{2_4_1_steerer_clamp.svg}

\subsection{For non e-bike builds}

If your headset has developed some play, act as follows: Preload the
long head tube headset with the star fangled nut (installed on
complete bikes, included with framesets) and the M6 topcap bolt, until
there is no more play in the headset.

\subsection{For Mahle X35 e-bike builds}

At the bottom of the steerer, pull on the cable, then unplug the iWoc
One and remove it. Now preload the long head tube headset with a long
threadbar with nuts and washers until there is no more play in the
headset. You may also use a standard steerer expander (not included
with the bike/frame) to adjust headset play. Reassemble the iWoc One
assembly and go ride.

\bodyimage{2_4_2_threadbar.svg}

\subsection{Play in your headset}

If your headset has developed some play, act as follows: Losen the
steerer clamp bolt. Tighten the short headtube topcap bolt until there
is no more play in the headset. Then tighten the steerer clamp bolt to
4.5 Nm.

\bodyimage{2_5_short_steerer_headset_preload.svg}

\subsection{Short headtube}

For the short headtube, only headsets with a block lock function can
be used. The maximum steering range for such headsets must not exceed
150°!

\bodyimage{2_6_1_headsets.svg}

\subsection{Long headtube}

For the long headtube: As an upper unit any standard 1 1/8” ahead
headset may be used.

\bodyimage{2_6_2_headsets.svg}

\subsection{Lower unit}

For the lower unit only original MONoPOLE parts must be used!

\bodyimage{2_6_3_headsets.svg}

\section{Cable routing}

\subsection{Fork: Brake hose routing}

Route the front brake hose as follows: Coming from the long steerer
tube fix it with the single cable stop. Then form an arc to the
underside of the steerer where you clamp it to its underside (on the
non drive side) using the fender mount bolt and cable clamp.

\bodyimage{3_1_1_brake_hose_routing.svg}

\subsection{Welded-on cable anchors}

Route it to the backside of the forkblades and secure it with a zip
tie in both welded-on cable anchors.

\bodyimage{3_1_2_brake_hose_routing.svg}

\subsection{Light cable routing}

When installing a lighting system, it is important to ensure that the
front light cable port is properly protected by the rubber grommet,
which should be securely in place to prevent dust and moisture from
entering. Excess cable should be pulled through to allow the full
steering range, as set by the BlockLock headset, without coming into
contact with the steering belt. To keep the cable securely in
position, it must be clamped to the underside of the steerer using the
fender mount bolt and a drive-side cable clamp. Proper installation
will ensure a clean setup while maintaining unrestricted steering and
overall safety.

\bodyimage{3_1_3_light_cable_routing.svg}

\subsection{Frame: Pinion bikes}

Ensure that the cable routing for the two shift cables is in the
correct order.

\bodyimage{3_2_1_brake_and_shift_cable_routing_pinion.svg}

\subsection{Cable routing pinion bikes}

\bodyimage{3_2_2_brake_and_shift_cable_routing.svg}

\subsection{Attention with pinion bikes}

Check that the cables running to the pinion gearbox do not interfere
with the rotating crank!

\bodyimage{3_2_5_brake_and_shift_cable_routing.svg}

\subsection{Brake and shift cable routing hub gear and derrailleur
  bikes}

\bodyimage{3_2_3_brake_and_shift_cable_routing.svg}

\subsection{Shift cable and rear brake hose}

\bodyimage{6_1_bio_frameset_rear_shift_cable_routing.svg}

\section{Racks/cargo}

\subsection{Steering range}

The full steering angle range must be available at all times, no
matter what cargo you have loaded to the bike.

\bodyimage{4_1_racks.svg}

\subsection{Check interference}

E.g. cargo must not interfere with your handlebars or any other
component of the steering mechanism—see section 2—or any other moving
part of the bicycle.

\bodyimage{4_2_racks.svg}

\subsection{Cargo securing items}

Cargo securing items need to be stowed when not in use. In a way that
they cannot losen or get in contact with any of the bikes moving
parts.

\bodyimage{4_3_racks_strap_good.svg}

\subsection{Cargo securing items}

\bodyimage{4_4_racks_strap_not_good.svg}

\section{Applicable only to No O1 electric}

\subsection{Mahle iWoc One and motor cable routing}

Make sure you stow the excess length of the iWoc One cable (for Mahle
X35 e-bike builds) inside the long head tube steerer and not inside
the frame main tube! Make sure that the iWoc One cable (for Mahle X35
e-bike builds) cannot interfere with the steering sprocket over the
whole range of steerer rotation.

\bodyimage{5_1_no_o1_electric_mahle_iwoc_one.svg}

\subsection{Motor cable fixation}

The motor cable must be secured using clamping bolts/clamps and must
not interfere with any moving parts, such as the cranks, disc rotor,
kickstand, etc.

\bodyimage{5_3_no_o1_electric_bolt_clamps.svg}

\subsection{Motor cable fixation}

The motor cable must not interfere with any moving parts, such as the
cranks, disc rotor, kickstand, etc.

\bodyimage{5_2_no_o1_electric_motor_cable.svg}

\subsection{Mahle motor, rear light and PAS (pedal assist sensor) cable routing}

The rear light cable and motor cable must not interfere with any
moving parts such as the belt ring, cranks, wheel, etc.

\bodyimage{5_4_no_o1_electric_sensor_motor_cable.svg}

\subsection{PAS sensor cable}

The PAS sensor cable must be secured with the designated bolt and must
not interfere with moving parts such as the belt ring, cranks, wheel,
etc.

\bodyimage{5_5_no_o1_electric_pas_sensor.svg}

\subsection{Turning the E-drive system on/off and changing settings.}

On the MONoPOLE No O1 electric, the on/off button is located on top of
the stem. The same button is used to switch between the support modes
of the drive system. For further information, please refer to the
Mahle X-35 E-drive system product manual you received when purchasing
the bike.

{\it *This information is mandated by governing bodies. We assume that
  you, as our customer, can figure out how to turn the system on and
  off. Additionally, our sales staff will have explained this to you
  during your purchase.}

\bodyimage{5_6_no_o1_electric_power_button.svg}

\section{Applicable only to No O1 bio and frameset}

\subsection{Rear brake and shift cable routing}

BSA adapter builds: The rear brake hose and shift cable enter the
adapter as shown and run through the drive-side and non-drive-side
chainstays, respectively, from there.

\bodyimage{6_1_bio_frameset_rear_shift_cable_routing.svg}

\subsection{BSA mount torque values}

BSA adapter: Make sure the bolts are tightened in alternation to a
value of: 10 Nm.

\bodyimage{6_2_bio_frameset_mount_torque_values.svg}

\section{Specs to respect}

\subsection{Max. brake rotor size}

\bodyimage{7_1_specs_to_respect_rotor.svg}

\subsection{Bottle cage bolt length}

In case you use longer bolts than originally supplied with your
bike/frame to mount a bottle cage, make sure bolts do not collide with
the steerer! With the bolts fully tight, check whether the handlebars
and steerer rotate freely.

\bodyimage{7_2_specs_to_respect_bottle.svg}

\subsection{Long steerer tube cable stop bolts}

Before every ride: Make sure the cable stop bolts do not collide with
the steerer! With the bolts fully tight, check whether the handlebars
and steerer rotate freely.

\bodyimage{7_3_specs_to_respect_cable_stops.svg}

\section{Cleaning of the bike/frameset}

Regular cleaning and maintenance are important for the safety and
longevity of your bike. We recommend wetting your bike only with a
standard water hose, using a gentle water jet. Avoid pointing the jet
directly at bearings or other delicate areas of the bike. Do not use
pressure washers! Clean your bike only with a clean, moist, and soft
towel. If needed, use a bicycle-specific cleaner.

{\it *This information is mandated by governing bodies. We assume that
  you, as our customer, are aware of the requirements for safely
  cleaning a bicycle, but it may still be beneficial to have a
  reminder.}

\end{document}

